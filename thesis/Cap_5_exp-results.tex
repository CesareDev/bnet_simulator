\chapter{Experimental results}
\label{chap:results}

\begin{comment}
This chapter presents and interprets the **results produced by the simulator**.

- **Baseline behavior**
    
    - Degradation of B-PDR under static beaconing as density increases.
        
    - Growth of collision rates and airtime saturation.
        
- **Effectiveness of ADAB**
    
    - Near-constant B-PDR across densities.
        
    - Strong divergence from static beaconing at high node counts.
        
- **Collision analysis**
    
    - Direct relationship between neighbor count and collision probability.
        
    - How adaptive throttling stabilizes channel occupancy.
        
- **Non-ideal channel results**
    
    - Absolute performance reduction under realistic over-water conditions.
        
    - Persistence of relative gains with ADAB.
        
- **Energy implications**
    
    - Analytical derivation of transmit energy.
        
    - Quantitative energy savings enabled by adaptive beaconing.
        
- **Discussion**
    
    - Why simple local adaptation is sufficient.
        
    - Trade-offs between latency, reliability, and energy.
        

The results are framed as **evidence that the simulator enables insights that would be infeasible with small-scale experiments alone**.
\end{comment}
% Utlizing the comments above try to write the chapter. And also explain the tecnical terms included. Maybe with online references usign LaTeX formatting and commands. you can use also the paper.pdf