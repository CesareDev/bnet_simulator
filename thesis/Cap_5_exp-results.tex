\chapter{Experimental Results}
\label{chap:results}

In this chapter we presents the experimental evaluation of ProSafe and the ADAB protocol, using the custom simulator described in Chapter~\ref{chap:implementation}. We begin by defining the metrics and experimental setup, then present results comparing static and adaptive beaconing under both ideal and realistic channel. We analyze the relationship between the node density, collision rates, and broadcast reliability, finally we discuss the energy implications and the insights enabled by simulation based evaluation.

\section{Performance Metrics}

The primary metric for evaluating broadcast based proximity detection systems is the reliability of beacon propagation across the network.

\subsection{Broadcast Packet Delivery Ratio (B-PDR)}

The Broadcast Packet Delivery Ratio (B-PDR) measures the fraction of intended beacon receptions that are actually successful. For each transmitted beacon $i$, we define:

\begin{itemize}
    \item $n_{\text{rcv},i}$: Number of receivers that successfully obtained the beacon
    \item $n_{\text{intended},i}$: Number of nodes within communication range at the time of transmission
\end{itemize}

The overall B-PDR across all $N_{\text{tx}}$ transmissions is:

$$\text{B-PDR} = \frac{1}{N_{\text{tx}}} \sum_{i=1}^{N_{\text{tx}}} \frac{n_{\text{rcv},i}}{n_{\text{intended},i}}$$

Aggregating across all transmissions:

$$\text{B-PDR} = \frac{\sum_{i=1}^{N_{\text{tx}}} n_{\text{rcv},i}}{\sum_{i=1}^{N_{\text{tx}}} n_{\text{intended},i}}$$

B-PDR reflects the probability that a beacon reaches its intended receivers, which is the fundamental reliability requirement for proximity based safety systems. A higher B-PDR means more reliable detection and reduced risk of missed proximity alerts.

\subsection{Collision Rate}

The collision rate measures the fraction of intended receptions that fail due to temporal and spatial overlap of concurrent transmissions:

$$\text{Collision Rate} = \frac{n_{\text{collided}}}{\sum_{i=1}^{N_{\text{tx}}} n_{\text{intended},i}}$$

where $n_{\text{collided}}$ is the total number of receptions lost due to collisions detected by the channel.

Collision rate directly indicates MAC layer contention: higher rates signal that multiple nodes are attempting to transmit simultaneously, causing interference. In broadcast networks without acknowledgments, collisions are cause of reliability degradation.

\subsection{Average Neighbor Count}

To understand the relationship between local density and protocol behavior, we track the average number of neighbors per node:

$$\bar{N}_{\text{neighbors}} = \frac{1}{N_{\text{buoys}}} \sum_{j=1}^{N_{\text{buoys}}} |\{k : d(j,k) \leq R_{\text{max}}\}|$$

where $N_{\text{buoys}}$ is the number of active nodes, and $d(j,k)$ is the distance between nodes $j$ and $k$.

This metric provides context for interpreting B-PDR and collision rate: as average neighbor count increases, we expect more contention and potentially lower reliability unless the protocol adapts appropriately.

\section{Experimental Setup}

All experiments were conducted using the custom simulator with parameters calibrated from the sea trial measurements.

\subsection{Simulation Parameters}

\textbf{Network configuration:}
\begin{itemize}
    \item Simulation area: 800 $\times$ 800 meters
    \item Duration: 600 seconds (10 minutes)
    \item PHY data rate: 1 Mbps (IEEE 802.11b DSSS~\cite{knowledgeshareKnowledgeShare})
    \item Beacon size: Approximately 500 bytes (depends on the neighbor list length)
\end{itemize}

\textbf{Channel model:}
\begin{itemize}
    \item High-probability range: 0-70 m ($P_{\text{rx}} = 0.9$)
    \item Low-probability range: 70-120 m ($P_{\text{rx}} = 0.15$)
    \item Beyond range: $>$120 m ($P_{\text{rx}} = 0$)
    \item Speed of light: $3 \times 10^8$ m/s (for propagation delay)
\end{itemize}

\textbf{MAC parameters:}
\begin{itemize}
    \item DIFS: 50 $\mu$s
    \item Slot time: 20 $\mu$s
    \item Contention window: CW = 16 (fixed, no exponential backoff)
    \item No RTS/CTS or ACK frames
\end{itemize}

\textbf{Protocol parameters:}
\begin{itemize}
    \item Static Beaconing Protocol (SBP): $BI_{\text{static}} = 0.25$ seconds
    \item ADAB: $BI_{\min} = 0.25$ s, $BI_{\max} = 5.0$ s, $N_{\text{thr}} = 10$ neighbors
    \item Neighbor timeout: 3 $\times BI_{\text{static}}$
    \item Jitter: $\pm 5\%$ random variation
\end{itemize}

\subsection{Experimental Scenarios}

\subsubsection{Population Scaling}

The primary experiment varies the total number of buoys from 20 to 300 in steps of 20, creating 15 distinct density scenarios. For each density level:

\begin{itemize}
    \item All buoys are randomly positioned within the simulation area
    \item A small percentage (20\%) are mobile with random velocities
    \item Stochastic churn adds and removes buoys throughout the simulation to model dynamic arrivals and departures
    \item Each scenario is repeated 15 times with different random seeds
\end{itemize}

This range spans from sparse coastal scenarios (20 buoys over 640,000 $\text{m}^2$) to crowded environments (300 buoys), enabling observation of protocol behavior across various density conditions.

\subsubsection{Churn Process}

To model realistic user dynamics, a stochastic churn process modifies the active population throughout the simulation:

\begin{itemize}
    \item Every 15-20 seconds (randomly), an update occurs
    \item With 50\% probability: remove up to 40\% of current buoys (respecting minimum threshold)
    \item Otherwise: add up to 40\% more buoys (respecting maximum capacity)
    \item Population is bounded: minimum = $\max(3, 0.2 \times N_{\text{total}})$, maximum = $N_{\text{total}}$
\end{itemize}

\subsection{Statistical Analysis}

Each density level is simulated 15 times with different random seeds, producing 15 independent samples of B-PDR, collision rate, and other metrics. Results are reported as:

\begin{itemize}
    \item Mean value across the 15 runs
    \item Error bars showing $\pm 1$ standard deviation
    \item Statistical significance assessed by checking if error bar intervals overlap
\end{itemize}

\section{Results: Ideal Channel}

We first present results under the ideal channel assumption (all transmissions within range succeed unless they collide), which isolates the impact of MAC layer contention from physical layer effects.

\subsection{Broadcast Packet Delivery Ratio}

Figure~\ref{fig:bpdr-ideal} shows B-PDR versus total number of buoys for both Static Beaconing Protocol (SBP) and ADAB under ideal channel conditions.

\begin{figure}[H]
\centering
\includegraphics[width=0.8\textwidth]{img/bpdr_ideal.png}
\caption{Broadcast packet delivery ratio (B-PDR) versus total number of buoys under ideal channel conditions.}
\label{fig:bpdr-ideal}
\end{figure}

The SBP (blue bars) shows a monotonic decline in B-PDR from approximately 0.95 at 20 nodes to around 0.70 at 300 nodes. This represents a 25\% reduction in reliability, meaning roughly one in four intended beacon receptions fail in high density scenarios. In contrast, ADAB (green bars) maintains a nearly constant B-PDR of 0.97-0.98 across the entire density range.

The average neighbor count (red curve, right axis) increases almost linearly from $\sim$3 at 20 nodes to $\sim$19 at 300 nodes. The degradation of SBP tracks this growth, confirming that local contention drives the performance decline.

\subsection{Collision Rate Analysis}

Figure~\ref{fig:collision-rate} shows how collision rate varies with node population for both protocols.

\begin{figure}[H]
\centering
\includegraphics[width=0.8\textwidth]{img/collision_rate.png}
\caption{Collision rate versus total number of buoys under ideal channel conditions.}
\label{fig:collision-rate}
\end{figure}

The collision rate for static beaconing increases almost linearly with population, from approximately 0.02 (2\% of receptions collide) at 20 nodes to 0.31 (31\%) at 300 nodes. ADAB maintains a collision rate below 0.04 (4\%) across the entire range. The collision rate remains nearly flat despite the increase in population. 

The collision rate directly explains the B-PDR results. SBP's rising collision rate impacts its reliability. ADAB's stable collision rate enables its consistent reliability.

\section{Results: Non Ideal Channel}

To assess performance under realistic conditions, we repeat the experiments with the probabilistic channel model calibrated from sea trial measurements.

\subsection{Broadcast Packet Delivery Ratio with Path Loss}

Figure~\ref{fig:bpdr-realistic} shows B-PDR versus population under the non-ideal channel.

\begin{figure}[H]
\centering
\includegraphics[width=0.8\textwidth]{img/bpdr_non_ideal.png}
\caption{Broadcast packet delivery ratio under non ideal channel.}
\label{fig:bpdr-realistic}
\end{figure}

Both protocols achieve significantly lower B-PDR values than under ideal conditions. SBP ranges from approximately 0.39 at 16-20 nodes down to 0.30 at 296-300 nodes. ADAB maintains 0.40-0.42 across the range. Despite lower absolute values, the relative benefit of ADAB remains clear. At high density, ADAB achieves about 0.10 absolute improvement (roughly 30\% relative) compared to SBP.
    
The non ideal channel introduces both collision losses (as before) and probabilistic losses (due to distance reception failure) meaning: a beacon that survives collision may still fail due to probabilistic loss, and vice versa.

\section{Discussion}

The experimental results provide several insights into the effectiveness of density aware adaptation and the role of simulation in protocol evaluation.

\subsection{Sufficiency of Local Information}

A remarkable aspect of ADAB is its simplicity: it uses only the count of neighbors heard in recent beacons as the only metric. Despite this simplicity, ADAB achieves substantial performance improvements across wide range of densities.

This demonstrates that, for broadcast based proximity detection in infrastructure free networks: global coordination or centralized scheduling is not always mandatory and neighbor count alone provides sufficient information to adapt transmission behavior in an effective way.

The success of this simple approach can be attributed to the direct relationship between neighbor count and collision probability in broadcast networks. More neighbors implies more potential simultaneous transmitters, which directly increases collision risk. By responding to this, ADAB implicitly adapts to the underlying congestion.

\subsection{Trade offs: Latency vs. Reliability vs. Energy}

ADAB makes explicit trade offs between competing objectives: in sparse scenarios short intervals are prioritize for rapid discovery, accepting higher energy consumption. This is appropriate when collision risk is low and quick detection is safety critical. In a dense scenario long intervals prioritize reliability and energy efficiency, accepting higher latency. This is appropriate when collision risk is high but many other nodes are already providing awareness.

Importantly, even the maximum interval (5.0s) remains within acceptable bounds for proximity safety: a vessel moving at 10 m/s covers only 50 meters in 5 seconds, well within the detection range. The increased latency is a reasonable trade off for the reliability and energy benefits.

\subsection{Scalability Limits}

While ADAB maintains high B-PDR across the tested range (20-300 nodes), there are practical limits. Even with maximum intervals, if density becomes extremely high, total channel airtime will eventually approach 100\% and performance must degrade, very high neighbor counts increase beacon size that eventually impacts transmission time and collision risk.

For the coastal safety application, densities beyond 300 nodes in an 800$\times$800m area are unlikely in realistic scenarios. However, the simulator enables exploration of these extreme cases to understand protocol limits.

\subsection{The Value of Simulation}

The experimental evaluation demonstrates the essential role of simulation in wireless protocol research: first of all physical experiments with 300 concurrent nodes are expensive and really complex to deploy and manage. Even if feasible, controlling and reproducing specific density scenarios would be challenging. Simulation enables exploration of a wide range of densities and conditions in a controlled, repeatable manner.

The simulator serves as a scientific instrument: it provides a reproducible, controlled platform for testing and performance evaluation across conditions that go from small  deployments to large scale recreational beach scenarios.

\section{Summary}

This chapter has presented comprehensive experimental results evaluating the ProSafe system and ADAB protocol. These results validate both the effectiveness of the ADAB protocol for proximity based safety applications and the utility of the simulator as a tool for protocol development and evaluation. The next chapter concludes the thesis with a summary of contributions, reflections on limitations, and directions for future research.