% Put in this section all the background knowledge needed to understand what you did.
\chapter{Background}
\label{chap:background}

\begin{comment}
This chapter provides the **theoretical and technological foundations** needed to understand the thesis.

- **Marine and coastal safety technologies**
    
    - Limitations of AIS, EPIRBs, PLBs, satellite-based systems.
        
    - Why these are unsuitable for continuous proximity awareness.
        
- **Wi-Fi for proximity detection**
    
    - Prior work on Wi-Fi-based discovery and passive monitoring.
        
    - Differences between infrastructure-based vs infrastructure-free settings.
        
- **802.11 MAC fundamentals**
    
    - Broadcast transmission, DCF, lack of ACKs, collision behavior.
        
    - Why broadcast reliability is the correct performance focus.
        
- **Beaconing strategies**
    
    - Static beaconing and its scaling limits.
        
    - Overview of adaptive beaconing in ad-hoc and vehicular networks.
        
- **Simulation in wireless research**
    
    - Role of simulators in exploring scalability, density, and protocol behavior.
        
    - Limitations of generic simulators for niche scenarios like over-water Wi-Fi broadcast.
        

This chapter justifies **why a custom simulator is needed** and why existing tools are insufficient.
\end{comment}
% Utlizing the comments above try to write the chapter. And also explain the tecnical terms included. Maybe with online references usign LaTeX formatting and commands. you can use also the paper.pdf