\begin{abstract}

Maritime proximity detection faces a critical safety gap: swimmers, kayakers, and small craft remain invisible to the larger ones because existing technologies (AIS, cellular tracking, satellite beacons) are either too expensive, infrastructure dependent, or designed for different use cases. Wi-Fi technology, offers a promising infrastructure free approach through periodic broadcast beaconing.

This thesis presents the design and implementation of a custom simulator developed for evaluating MAC layer broadcast protocols in these scenarios with dynamic populations. The simulator implements IEEE 802.11 DCF behavior including carrier sensing, DIFS waiting, random backoff, and collision detection, calibrated using empirical measurements from sea trials with Raspberry Pi prototypes.

To demonstrate the simulator's capabilities, we evaluate the ProSafe (Proximity based Safety) system and its Adaptive Density Aware Beaconing (ADAB) protocol. ProSafe is a Wi-Fi based maritime safety network where smart buoys broadcast periodic location aware beacons to enable vessels to detect people in the water. Through systematic experiments, we demonstrate that fixed interval Static Beaconing Protocol (SBP) suffers reliability degradation as density increases, with Broadcast Packet Delivery Ratio (B-PDR) declining. On the other hand, ADAB maintains consistent reliability across the entire density range by dynamically adjusting beacon intervals based on local neighbor counts.

These findings demonstrate that simple, locally observable information is sufficient for effective adaptation without requiring global coordination, complex channel measurements, or infrastructure support.

This work contributes both a validated simulation tool for MAC layer protocol research and scientific insights into the sufficiency of local adaptation mechanisms for managing channel access in infrastructure free broadcast networks. The simulator provides a practical platform for future research on adaptive beaconing protocols, enabling rapid prototyping and evaluation before costly physical deployments.

\end{abstract}