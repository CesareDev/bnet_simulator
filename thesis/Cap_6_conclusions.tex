\chapter{Conclusions}
\label{chap:conclusion}

This thesis presented the design, implementation, and evaluation of a custom event driven simulator for infrastructure free Wi-Fi broadcast networks, with application to proximity based safety systems. This chapter summarizes the contributions, acknowledges limitations, and outlines directions for future research.

\section{Summary of Contributions}

The primary contribution of this thesis is the development of a custom simulator designed for evaluating MAC layer broadcast protocols in infrastructure free wireless networks with dynamic populations. This work makes several distinct contributions:

\subsection{Simulator Design and Implementation}

We designed and implemented a discrete event simulator. The architecture cleanly separates the simulation core (event queue, time progression), physical layer (channel model, propagation), MAC layer (CSMA/CA, collision detection), and protocol logic (beaconing strategies).

The simulator implements IEEE 802.11 DCF~\cite{dcf} behavior including carrier sensing, DIFS waiting, random backoff, and collision detection for broadcast transmissions. Unlike purely analytical models, the simulator's wireless channel is calibrated using empirical measurements from sea trials with Raspberry Pi based prototypes grounding simulation results in physical reality.

The simulator implements node churn (arrivals and departures) to model realistic deployment scenarios where users enter and leave the system throughout the observation period. This capability is important for evaluating protocol responsiveness to changing density conditions.

The implementation tracks multiple performance indicators including Broadcast Packet Delivery Ratio (B-PDR), collision rates, per node discovery counts. This data collection enables the analysis of protocol behavior from different perspectives.

The modular architecture supports extension with new beaconing protocols (demonstrated by implementing SBP, ADAB), alternative channel models, different mobility patterns without requiring modifications to core components.

\subsection{Protocol Evaluation: The ProSafe Case Study}

To demonstrate the simulator's utility and validate its design, we used it to evaluate the ProSafe proximity safety system and its Adaptive Density-Aware Beaconing (ADAB) protocol: through systematic experiments varying node populations from 20 to 300, we evaluate the scalability limitations of fixed interval beaconing and demonstrated ADAB's ability to maintain consistent reliability across this range. Static beaconing shows B-PDR degradation from 95\% to 70\% as density increases, while ADAB maintains 97-98\% throughout.

We identified collision probability as the primary factor limiting scalability in broadcast networks and showed how density aware interval adaptation bounds collision rates (below 4\% for ADAB vs. 31\% for static beaconing at high density) by considering local channel occupancy. The evaluation revealed that adaptive beaconing achieves both improved reliability and reduced energy consumption in high density scenarios. ADAB reduces transmission energy by up to 95\% in dense environments while maintaining or improving delivery ratios.

The results demonstrate that simple, locally observable information (neighbor count) is sufficient for effective adaptation. Complex channel measurements, global coordination, or sophisticated signal processing are unnecessary for achieving substantial performance gains.

\subsection{Methodological Contribution}

Beyond the simulator and protocol, this work contributes a methodological approach: by calibrating the channel model from physical measurements and validating protocol behavior against analytical models in simple cases, we demonstrate how simulation can fill the gap between pure analysis and full scale deployment.

Rather than attempting to model every aspect of wireless communication, the simulator focuses on the phenomena essential to the research question (MAC layer contention, broadcast collisions, density effects).

All simulation parameters are externalized in configuration files, code is modularly organized. This approach facilitates reproducibility and enables other researchers to validate results or extend the work.

\section{Scientific Insights}

\subsection{Broadcast Scalability in Infrastructure Free Networks}

The experimental results clearly demonstrate that fixed parameter protocols do not scale well to varying density conditions in infrastructure free broadcast networks. This creates a need for an adaptation mechanisms that use locally observable metrics. The success of ADAB's simple neighbor count based adaptation suggests that such proxies can be highly effective despite their simplicity.

\subsection{Sufficiency of MAC Layer Adaptation}

MAC layer adaptation alone without requiring changes to physical layer, routing protocols, or application logic can substantially improve performance. By operating entirely at the beaconing/scheduling layer and leaving the underlying CSMA/CA mechanism unchanged, ADAB achieves its benefits with minimal implementation complexity and no hardware modifications.

This suggests that for many wireless protocol challenges, the solution may not require sophisticated cross layer optimization or redesign. Simple, focused interventions at the appropriate layer can be highly effective.

\subsection{Local Information vs. Global Coordination}

The contrast between ADAB (using only local neighbor counts) and hypothetical centralized scheduling schemes is instructive. ADAB's effectiveness demonstrates that global coordination is not necessary for managing channel access in moderate scale ad hoc networks. Local information, when properly used, enables nodes to implicitly coordinate their behavior without message passing or infrastructure.

This has important implications for system robustness: local adaptation continues to function even when network partitions occur or environmental conditions change rapidly.

\section{Limitations}

This work has several limitations that should be acknowledged:

\subsection{Simplified Physical Layer}

The channel model uses a piecewise constant probability function based on distance, which captures average behavior but abstracts away many real world effects: short term signal fluctuations due to wave motion, reflections, and interference are not modeled explicitly, the ability of receivers to decode the stronger of two overlapping signals is not considered; all collisions result in complete packet loss. All transmissions use 1 Mbps regardless of distance or channel quality, while real systems might use rate adaptation

These simplifications are justified for the thesis scope, which is not focused on physical layer phenomena but rather on MAC layer adaptation, but would need to be addressed for studies focused on physical layer phenomena.

\subsection{Mobility Modeling}

The mobility model uses simple kinematic motion with boundary reflection. Real water activities involve more complex patterns:

\begin{itemize}
    \item Ocean currents and tidal flows create correlated movement.
    \item Swimmers and kayakers follow non random paths (e.g., parallel to shore).
    \item Wave action creates vertical motion and temporary line of sight obstructions.
    \item Groups of users may cluster spatially.
\end{itemize}

More realistic mobility traces from GPS tracking of actual water activities would improve the validity of results, though the fundamental scalability insights would likely persist.

\subsection{Limited Scale Range}

While the simulator was tested up to 300 nodes in an 800$\times$800m area, even larger scales (thousands of nodes, or dense deployment in smaller areas) may reveal different behavior or additional scalability limits.

\subsection{Energy Model Completeness}

The energy analysis focused on transmission energy, which is appropriate for evaluating the direct impact of beacon rate changes. However, a complete energy model would include:

\begin{itemize}
    \item Reception energy (listening to channel and decoding beacons)
    \item Idle listening energy (monitoring channel when not transmitting or receiving)
    \item Computational energy (protocol logic, GPS, neighbor table management)
    \item Peripheral energy (sensors, displays, logging)
\end{itemize}

The relative importance of these factors depends on hardware characteristics and implementation efficiency.

\section{Future Work}

Several promising directions emerge from this thesis work, building upon both the simulator capabilities and the insights gained from the ProSafe evaluation.

\subsection{Enhanced Channel Model with Packet Length Dependent Loss}

The current probabilistic channel model uses distance based reception probabilities that are independent of packet size. In reality, longer packets have higher probability of loss due to: increased exposure time to interference and fading, higher likelihood of bit errors accumulating beyond error correction capabilities, and greater susceptibility to varying channel conditions during transmission

A more realistic model could incorporate packet length dependent loss probability:

$$P_{\text{loss}}(d, L) = P_{\text{base}}(d) \cdot \left(1 + \alpha \cdot \frac{L}{L_{\text{ref}}}\right)$$

where $P_{\text{base}}(d)$ is the baseline loss probability at distance $d$, $L$ is the packet length in bytes, $L_{\text{ref}}$ is a reference packet size, and $\alpha$ is a scaling factor calibrated from measurements.

This enhancement would enable investigation of important trade offs: ADAB beacons grow with neighbor count (28 bytes per neighbor). Should nodes limit neighbor list size in dense scenarios to improve delivery probability? Can nodes detect high loss rates and reduce beacon content? Should critical information be sent in multiple small beacons rather than one large beacon?

Currently, the simulator uses a variable beacon sizes, but the channel model treats all packets equivalently. Incorporating size dependent loss would provide more accurate modeling of real world performance.

\subsection{Adaptive Context-Aware Beaconing (ACAB) Evaluation}

The simulator implements an alternative protocol, Adaptive Context-Aware Beaconing (ACAB), that extends ADAB by considering multiple contextual factors beyond local density:

\begin{enumerate}
    \item \textbf{Density component:} Similar to ADAB, normalized neighbor count with threshold $N_{\text{thr}} = 10$
    
    \item \textbf{Contact recency component:} Time since last beacon reception, with threshold of 20 seconds. Nodes that have recently heard neighbors can beacon less frequently, while isolated nodes beacon more aggressively
    
    \item \textbf{Mobility component:} Node speed relative to expected velocity. High mobility nodes beacon more frequently to ensure neighbors track their changing position
\end{enumerate}

The combined adaptation score is computed as:

$$s_{\text{combined}} = 0.4 \cdot s_{\text{density}} + 0.3 \cdot s_{\text{contact}} + 0.3 \cdot (1 - s_{\text{mobility}})$$

This weighted combination is then mapped quadratically to the interval range as in ADAB. Future work should systematically evaluate ACAB against ADAB to determine: whether the additional complexity (velocity tracking, contact time management) provides measurable benefits, how weight parameters (0.4, 0.3, 0.3 in the formula) should be tuned for different scenarios and the computational and energy overhead of the more complex interval calculation.

\subsection{Multi Hop Awareness Mechanisms}

The simulator implements two distinct multi hop awareness modes that extend detection range beyond direct radio communication:

\subsubsection{Append Mode (Passive Multi Hop)}

In append mode, nodes include their neighbor list in each beacon. Receivers learn about additional nodes indirectly without any forwarding overhead:

\begin{itemize}
    \item Buoy A hears beacon from B containing C in B's neighbor list
    \item A adds C to its \texttt{discovered\_nodes} table even though A never directly heard C
    \item This provides two hop awareness: A knows about nodes that are neighbors of A's neighbors
    \item Effective for extending situational awareness when vessels monitor multiple proximity zones
\end{itemize}

The tradeoff is increased beacon size: each neighbor adds 28 bytes (UUID + timestamp + position). In dense scenarios, neighbor lists can grow large, potentially reducing delivery probability if packet length dependent loss is modeled.

\subsubsection{Forwarded Mode (Active Multi Hop)}

In forwarded mode, nodes explicitly retransmit received beacons to extend physical range:

\begin{itemize}
    \item Each beacon includes \texttt{origin\_id} (original sender) and \texttt{hop\_limit} (TTL counter)
    \item Upon reception with $\texttt{hop\_limit} > 0$, nodes create a forwarded copy with decremented hop limit
    \item Duplicate detection (via origin ID and timestamp) prevents forwarding loops
    \item Forwarding occurs after random delay using the same CSMA/CA mechanism
    \item Maximum hop limit is configurable (default 2 hops)
\end{itemize}

Forwarded mode significantly increases channel load: each beacon can trigger multiple retransmissions. However, it extends effective range for scenarios where line of sight obstructions cause intermittent connectivity.

\subsubsection{Comparison between Append mode and Forward mode}

During the testing of these two modes, we observed what it was described in the previous subsections. The append mode is not impacting the channel load too much, since no additional transmissions are generated. However, the size of the beacons increases with the number of neighbors, which could impact the delivery ratio if a packet length dependent loss model is implemented. On the other hand, the forwarded mode increases significantly the channel load, since each beacon can generate multiple retransmissions. This could lead to higher collision rates in dense scenarios.

Following three plots of the preliminary results obtained with the normal mode (without multihop) and the multihope ones. The metrics used is the average percentage of discovered nodes varying the number of nodes in the scenario. The configuration of the simulation are the same as described in Chapter~\ref{chap:results}.

\begin{figure}[H]
    \centering
    \includegraphics[width=0.7\textwidth]{img/discovered_none.png}
    \caption{Normal Mode - Average Percentage of Discovered Nodes vs Number of Nodes}
    \label{fig:normal_mode_discovery}
\end{figure}

\begin{figure}[H]
    \centering
    \includegraphics[width=0.7\textwidth]{img/discovered_forward.png}
    \caption{Forward Mode - Average Percentage of Discovered Nodes vs Number of Nodes}
    \label{fig:forward_mode_discovery}
\end{figure}

\begin{figure}[H]
    \centering
    \includegraphics[width=0.7\textwidth]{img/discovered_append.png}
    \caption{Append Mode - Average Percentage of Discovered Nodes vs Number of Nodes}
    \label{fig:append_mode_discovery}
\end{figure}

In the plots above is included also the new protcol ACAB described in the previous subsection which is still in development and needs further improvement. From the preliminary results we can see that with the forward mode (Figure~\ref{fig:forward_mode_discovery}) we have an improvement in the discovery rate with respect to the normal mode (Figure~\ref{fig:normal_mode_discovery}), but with the append mode (Figure~\ref{fig:append_mode_discovery}) the discovery rate is even higher. This probably comes from the fact that in the append mode there are no additional transmissions that could increase the collision rate, while in the forward mode the additional transmissions lead to more collisions, especially in dense scenarios as we can see in the following plot.

\begin{figure}[H]
    \centering
    \includegraphics[width=0.7\textwidth]{img/collision_forward.png}
    \caption{Forward Mode - Collision Rate vs Number of Nodes}
    \label{fig:collision_rate_forward}
\end{figure}

\subsubsection{Future Multi Hop Research Directions}

Comprehensive evaluation of these implemented modes should: measure effective detection radius improvement vs. direct transmission only, quantify how forwarded beacons increase collision probability in dense scenarios, analyze energy consumption impact of forwarding, and determine optimal hop limits for different deployment contexts.

The simulator's implementation of both modes (controllable via \texttt{multihop\_mode} configuration parameter) enables systematic experimental comparison of these strategies.

\subsection{Real Mobility Traces}

Collecting GPS traces from actual swimmers, kayakers, and small watercraft in coastal environments would enable: trace driven simulations with realistic mobility patterns including correlated movement from currents and waves, identification of mobility specific challenges (e.g., rapid approach scenarios, clustering effects) Such traces could be collected via smartphone apps or dedicated GPS loggers during supervised water activities.

\subsection{Application to Other Domains}

While developed for maritime safety, the simulator and insights are applicable to other infrastructure free broadcast scenarios: vehicular networks (emergency beacons between cars, bikes, pedestrians), disaster response (ad hoc communication when infrastructure is damaged), wildlife tracking, indoor positioning and IoT sensor networks.

\section{Broader Impact}

Beyond the technical contributions, this work has potential societal impact:

\subsection{Maritime Safety}

The ProSafe system addresses a real safety gap in coastal waters. Swimmers, snorkelers, kayakers, and divers currently have limited options for making themselves visible to vessels. By leveraging ubiquitous Wi-Fi hardware and requiring no infrastructure, ProSafe could provide an accessible, low cost safety enhancement for recreational water users.

The simulator enables evaluation of such systems before expensive pilot deployments, reducing the risk and cost of innovation in safety critical domains.

\section{Final Remarks}

This thesis has presented a custom event driven simulator for infrastructure free Wi-Fi broadcast networks and demonstrated its utility through the evaluation of the ProSafe maritime safety system and ADAB protocol. The work makes contributions at multiple levels:

\begin{itemize}
    \item \textbf{As a tool:} The simulator provides a reusable platform for evaluating broadcast protocols under varying density and mobility conditions
    
    \item \textbf{As a case study:} The ProSafe evaluation demonstrates that simple density aware adaptation can achieve substantial improvements in broadcast reliability, collision reduction, and energy efficiency
    
    \item \textbf{As a methodology:} The approach of calibrating simulation from physical measurements, focusing scope appropriately, and validating through multiple means offers a template for rigorous protocol evaluation
    
    \item \textbf{As an application:} ProSafe addresses a real safety need in coastal environments using accessible, low cost technology
\end{itemize}

The key insight is that MAC layer broadcast scalability in infrastructure free networks can be significantly improved through rather simple adaptation. Global coordination, channel measurements, or cross layer optimization are not necessary for substantial gains.

The simulator developed in this thesis is a tool applicable beyond ProSafe to any broadcast based proximity or discovery protocol. By providing controlled, reproducible evaluation at scales infeasible for physical experiments, it enables exploration of the various design choices and comparison of alternatives.

In conclusion, this thesis demonstrates that simulators grounded in physical measurements and validated against analytical models are useful scientific instruments for protocol research. When combined with protocol design, they enable development of practical systems that scale from small deployments to large scale real world scenarios.