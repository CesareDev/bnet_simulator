\chapter{Implementation}
\label{chap:implementation}

\begin{comment}
    This chapter details the **simulator itself**, which is the **core thesis contribution**.

- **Simulator architecture**
    
    - Event-driven design (transmissions, receptions, mobility, churn).
        
    - Modular separation of PHY, MAC, protocol logic, and metrics.
        
- **Node model**
    
    - Buoy and vessel abstractions.
        
    - Neighbor tables, TTL handling, and rebroadcast logic (TTL=1).
        
- **MAC-layer modeling**
    
    - DCF behavior, contention window, back-off, broadcast collisions.
        
    - Timing model for airtime and overlapping transmissions.
        
- **Protocol implementation**
    
    - SBP: fixed-interval logic.
        
    - ADAB: neighbor counting, quadratic interval mapping, jitter.
        
- **Calibration integration**
    
    - How real-world measurements are injected into the simulator.
        
- **Extensibility**
    
    - How the simulator supports future protocols (e.g., multi-hop, different PHYs).
        
    - Why it is reusable beyond ProSafe.
        

This chapter **proves engineering depth** and distinguishes the thesis from the shorter paper.
\end{comment}
% Utlizing the comments above try to write the chapter. And also explain the tecnical terms included. Maybe with online references usign LaTeX formatting and commands. you can use also the paper.pdf
