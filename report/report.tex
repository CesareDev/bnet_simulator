\documentclass{article}
\usepackage{algorithm}
\usepackage{algpseudocode}
\usepackage{amsmath}

\begin{document}

\title{Channel Class Pseudocode for Beacon Network Simulator}
\section{Dynamic Beacon Scheduling Algorithm}

The interval $I$ between consecutive beacon transmissions is computed as:

\begin{equation}
I = I_{\text{base}} - I_{\text{contact}} + I_{\text{density}}
\end{equation}

\noindent where:

\begin{align}
I_{\text{base}} &= I_{\text{min}} + 0.7 \cdot (I_{\text{max}} - I_{\text{min}}) \\
I_{\text{contact}} &= 0.5 \cdot C_s \cdot (I_{\text{base}} - I_{\text{min}}) \\
I_{\text{density}} &= 0.6 \cdot D_f \cdot (I_{\text{max}} - I_{\text{base}})
\end{align}

The contact score $C_s$ measures recency of neighbor communication:
\begin{equation}
C_s = \max\left(0, 1 - \frac{\Delta t}{20}\right)
\end{equation}
where $\Delta t$ is time (in seconds) since most recent neighbor contact. The density factor $D_f$ reflects local network congestion:
\begin{equation}
D_f = \min\left(1, \frac{N}{15}\right)
\end{equation}
where $N$ is the number of neighbors in communication range. The finale interval $I_{\text{final}}$ is computed as:

\begin{equation}
I_{\text{final}} = \max(I_{\text{min}}, \min(I, 0.9 \cdot (I_{\text{max}} - I_{\text{min}})))
\end{equation}

\end{document}